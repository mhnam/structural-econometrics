
\documentclass[11pt]{article}
\usepackage{lscape}
\usepackage{amsfonts}
\usepackage{geometry}
\usepackage{amsmath}
\usepackage{amssymb}

\setcounter{MaxMatrixCols}{10}
\newtheorem{theorem}{Theorem}
\newtheorem{acknowledgement}[theorem]{Acknowledgement}
\newtheorem{algorithm}[theorem]{Algorithm}
\newtheorem{axiom}[theorem]{Axiom}
\newtheorem{case}[theorem]{Case}
\newtheorem{claim}[theorem]{Claim}
\newtheorem{conclusion}[theorem]{Conclusion}
\newtheorem{condition}[theorem]{Condition}
\newtheorem{conjecture}[theorem]{Conjecture}
\newtheorem{corollary}[theorem]{Corollary}
\newtheorem{criterion}[theorem]{Criterion}
\newtheorem{definition}[theorem]{Definition}
\newtheorem{example}[theorem]{Example}
\newtheorem{exercise}[theorem]{Exercise}
\newtheorem{lemma}[theorem]{Lemma}
\newtheorem{notation}[theorem]{Notation}
\newtheorem{problem}[theorem]{Problem}
\newtheorem{proposition}[theorem]{Proposition}
\newtheorem{remark}[theorem]{Remark}
\newtheorem{solution}[theorem]{Solution}
\newtheorem{summary}[theorem]{Summary}
\newenvironment{proof}[1][Proof]{\noindent\textbf{#1.} }{\ \rule{0.5em}{0.5em}}

\geometry{left=1.3in, right=1.3in, top=1.2in, bottom=1.2in}

\begin{document}

\begin{flushright}
Econometrics III 47-813

Spring 2026

Robert A. Miller
\end{flushright}

\begin{center}
\textbf{ASSIGNMENT 1 (Kang et al, 2015)}
\end{center}

\vspace{0.5cm}

\noindent \textbf{Overview.}
This assignment concerns Conditional Choice Probability (CCP) estimators and their application to regulatory decision-making. 
You will replicate and extend the analysis in Kang, Lowery, and Wardlaw (2015, \textit{Review of Financial Studies}), 
which studies the FDIC's bank closure decisions using a dynamic discrete choice framework.

\vspace{0.3cm}

\noindent \textbf{Instructions.}
Work in groups of about three. Each group should submit a single, self-contained report with well-documented code as an appendix. 
Hand written work will not be graded. The code must be clearly written with comments where appropriate so that a reader can easily follow. 
Poor grammar, unclear expression, and lack of precision, will be graded as if I have very limited expertise in
this area. All questions carry equal weight unless otherwise stated.

\vspace{0.5cm}

\noindent\textbf{Model}
The economic agent is the FDIC. Given a state variable $x_{it}$, the FDIC chooses $d_{it} \in \{0, 1\}$ where
\[
d_{it} =
\begin{cases}
0 & \text{if FDIC chooses to let the bank continue in period } t \\
1 & \text{if FDIC chooses to close the bank in period } t.
\end{cases}
\]

\vspace{0.3cm}
\noindent\underbar{Per-period payoffs}
The per-period payoff to the FDIC is:
\[
u_j(x_{it}, \varepsilon_{it}) = -\mathbf{1}_{d_{jt}=1} c(x_{it}) + \varepsilon_{jit} =
\begin{cases}
\varepsilon_{0it} & \text{if } d_{it} = 0 \\
-c(x_{it}) + \varepsilon_{1it} & \text{if } d_{it} = 1
\end{cases}
\]
where $\varepsilon_{it} := (\varepsilon_{0it}, \varepsilon_{1it})$.

\vspace{0.3cm}
\noindent\underbar{Objective function}
The FDIC maximizes:
\[
(1 - d_{it}) u_0(x_{it}, \varepsilon_{it}) + d_{it} u_1(x_{it}, \varepsilon_{it}) + \sum_{s=t+1}^{T} \beta^{s-t} \mathbb{E}\left[(1 - d_{is}) u_0(x_{is}, \varepsilon_{is}) + d_{is} u_1(x_{is}, \varepsilon_{is}) \mid x_{it}, d_{it}\right].
\]

\vspace{0.3cm}
\noindent\underbar{Conditional value functions}
The conditional value function given choice $j$ at state $x_{it}$ is:
\begin{align*}
v_j(x_{it}) &= u_j(x_{it}) + \sum_{s=t+1}^{T} \beta^{s-t} \mathbb{E}\left[V(x_{i,t+1}, \varepsilon_{i,t+1}) \mid x_{it}, d_{it} = j\right], \quad j \in \{0, 1\}
\end{align*}
where $u_0(x_{it}) = 0$ and $u_1(x_{it}) = -c(x_{it})$.

\vspace{0.5cm}

\noindent\textbf{Question 1. }
Derive the Hotz-Miller inversion result under the assumption that $\varepsilon_{it}$ follows an i.i.d.\ 
Type-I extreme value distribution with mean 0 and scale parameter $\sigma$. Specifically, show that:
\[
v_0(x_{it}) - v_1(x_{it}) = \sigma \log \frac{p_0(x_{it})}{p_1(x_{it})}
\]
where $p_j(x_{it})$ denotes the conditional choice probability of action $j$ given state $x_{it}$.

\vspace{0.5cm}

\noindent\textbf{Question 2. }
Derive the following equation (6) in Kang et al.\ (2015) uses notation $p_1^{(1)}(x_{0it})$ and $c^{(1)}(x_{0it})$:
\[
\sigma \log \frac{p_0(x_{it})}{p_1(x_{it})} = \beta \mathbb{E}_x\left[ -\sigma log p_1^{(1)}(x_{0it}) - c^{(1)}(x_{0it}) \right] + c(x_{it})
\]
\begin{itemize}
    \item[(a)] Explain what these terms mean using the definition of $A^{(k)}(x_{jit})$ in the paper (page 1067).
    \item[(b)] What exactly is the expectation operator in equation (6) doing?
\end{itemize}

\vspace{0.5cm}

\noindent\textbf{Question 3. }
Consider the cost function $c(x_{it})$. The authors separate monetary costs $MC(x_{it})$ and non-monetary costs 
(parameterized by $\tilde{x}_{it}'\theta_{nmc}$). Tables 3 and 7 present the parameter estimates for $\theta_{mc}$ 
and $\theta_{nmc}$.
\begin{itemize}
    \item[(a)] What are the ``independent variables'' for monetary costs and non-monetary costs?
    \item[(b)] What assumptions must be satisfied for these estimates to be identified?
    \item[(c)] Critically evaluate these identification assumptions. Do you find them plausible? Why or why not?
\end{itemize}

\vspace{0.5cm}

\noindent\textbf{Question 4. }
How do the authors estimate the transition processes for the state variables?
\begin{itemize}
    \item[(a)] Describe their approach in detail.
    \item[(b)] Considering the model structure, would you use a different approach? Justify your answer.
\end{itemize}

\vspace{0.5cm}

\noindent\textbf{Question 5. }
Replicate the baseline results from Kang et al.\ (2015). Use the replication code and data provided in the \texttt{tutorial1/replication/} folder.
\begin{itemize}
    \item[(a)] Run the first-stage logit estimation to obtain CCPs. Report the coefficient estimates and compare them to Table 2 in the paper.
    \item[(b)] Run the second-stage GMM estimation to recover structural parameters. Report the estimates for monetary and non-monetary cost parameters and compare them to Tables 3 and 7.
    \item[(c)] Briefly discuss any discrepancies between your results and the published estimates.
\end{itemize}
\vspace{0.5cm}

\noindent For the following set of questions (Questions 6 -- 8), modify the replication code to implement the alternative specifications. 
Clearly describe how you implement this modification in the replication code. For each changes you should make 
to the conditional choice probabilities, think about the functional form of monetary costs and non-monetary costs of
FDIC on closing a bank.
\vspace{0.5cm}

\noindent\textbf{Question 6. }
Include a time trend in the estimation of the choice probabilities.
\vspace{0.5cm}

\noindent\textbf{Question 7. }
Omit the effect of the House of Representatives (denoted as \texttt{House} in the paper) 
and/or Senate in the estimation of the choice probabilities.
\vspace{0.5cm}

\noindent\textbf{Question 8. }
Add squared terms on time, House, and/or Senate in the estimation of the choice probabilities.
\vspace{0.5cm}

\noindent\textbf{Question 9. }
Synthesize your findings from the alternative specifications.
\begin{itemize}
    \item[(a)] Which specification choices have the largest impact on the estimated cost of bank closure?
    \item[(b)] What are the key sources of identification in this model? How robust is identification to specification choices?
    \item[(c)] Based on your analysis, what policy conclusions can be drawn about FDIC behavior during the financial crisis?
\end{itemize}

\vspace{0.7cm}

\noindent\textbf{Deliverables.}
Submit a report that:
\begin{itemize}
    \item[(i)] Answers each question with appropriate mathematical derivations.
    \item[(ii)] Reports and interprets your empirical estimates across specifications.
    \item[(iii)] Discusses limitations and possible extensions.
\end{itemize}
Attach your code (well commented) as an appendix. Explain any modifications you made to the replication code.

\vspace{0.5cm}

\noindent\textbf{References.}

\noindent Hotz, V.J. and Miller, R.A. (1993). Conditional Choice Probabilities and the Estimation of Dynamic Models. \textit{The Review of Economic Studies}, 60(3):497--529.

\noindent Kang, A., Lowery, R., and Wardlaw, M. (2015). The Costs of Closing Failed Banks: A Structural Estimation of Regulatory Incentives. \textit{The Review of Financial Studies}, 28(4):1060--1102.

\end{document}
